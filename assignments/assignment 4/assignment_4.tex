\documentclass[fleqn, 12pt]{article}

% packages %
\usepackage[includeheadfoot,headheight=15pt,margin=0.5in,left=1in,right=1in,headsep=10pt]{geometry} % page margins %
\usepackage{mathtools, amssymb} % math %
\usepackage{tabularx, multirow} % tables %
\usepackage[outputdir=.latexcache]{minted} % code %
\usepackage{graphicx} % graphics %
\usepackage{enumerate} % lists %
\usepackage{adjustbox} % images %
\usepackage[T1]{fontenc} % fonts %
\usepackage[protrusion=true,expansion=true]{microtype} % font clarity %
\usepackage{fancyhdr} % headers and footers %
\usepackage{lastpage} % reference page numbers %
\usepackage{color} % colors for code %
\usepackage{tikz} % for graphs %
\usepackage{soul} % for strikethroughout %
\usepackage{upquote} % Fix single quotes %
\usepackage{etoolbox} % Conditional checks %
\usepackage[hidelinks]{hyperref} % Hyperlinks %
\usepackage{indentfirst} % fix indentation - only for essays %
\usepackage[figure,table]{totalcount} % For counting tables and figures %
\usepackage[utf8]{inputenc} % Encode as UTF-8 %
\usepackage{biblatex} % References %
\usepackage{tocloft} % Table of Contents dots %
\usepackage{setspace} % double space %
\addbibresource{references.bib} % bib source %

% Document details %
\newcommand{\university}{University of Ottawa}
\newcommand{\name}{Matthew Langlois}
\newcommand{\studentNumber}{7731813}
\newcommand{\semester}{Winter 2018}
\newcommand{\assignmentType}{Assignment}
\newcommand{\assignmentNumber}{4}
\newcommand{\dueDate}{April 9, 2018}
\newcommand{\courseCode}{GEG2320}
\newcommand{\courseTitle}{Introduction to Geomatics}
\newcommand{\essayTitle}{<Title>} % only for essays %
\newcommand{\essaySubtitle}{<subtitle>} % only for essays %
\newcommand{\essayAbstract}{} % Only for essays -- leave empty for no abstract %

% Center image and diagrams %
\adjustboxset*{center}

% Code settings %
\setminted{
    fontfamily=tt,
    gobble=0,
    frame=single,
    funcnamehighlighting=true,
    tabsize=4,
    obeytabs=false,
    mathescape=false
    samepage=false,
    showspaces=false,
    showtabs =false,
    texcl=false,
    breaklines=true,
}

% inline code %
\definecolor{codegray}{gray}{0.9}
\newcommand{\code}[2]{\colorbox{codegray}{\mintinline{#1}{#2}}}

% Code from tile %
\newcommand{\codefile}{\inputminted}

% Graphing stuff %
\usetikzlibrary{arrows.meta}
\usetikzlibrary{positioning}
\usetikzlibrary{matrix}
\usetikzlibrary{automata}

% Define ceiling and floor functions %
\DeclarePairedDelimiter\ceil{\lceil}{\rceil}
\DeclarePairedDelimiter\floor{\lfloor}{\rfloor}

% Create set compliment command %
\newcommand{\setcomp}[1]{{#1}^{\mathsf{c}}}

% Create logic command aliases %
\newcommand{\limplies}{\rightarrow}
\newcommand{\nequiv}{\not\equiv}
\newcommand{\liff}{\leftrightarrow}

% first page header & footer %
\fancypagestyle{assignment}{
    \fancyhf{}
    \renewcommand{\footrulewidth}{0.1mm}
    \fancyfoot[R]{\assignmentType\text{ }\ifdefempty{\assignmentNumber}{}{\#}\assignmentNumber}
    \fancyfoot[C]{\thepage \hspace{1pt} of \pageref{LastPage}}
    \fancyfoot[L]{\courseCode\text}
    \renewcommand{\headrulewidth}{0mm}
}

% Frontmatter for essay page numbering%
\fancypagestyle{frontmatter}{
    \fancyhf{}
    \renewcommand{\footrulewidth}{0.1mm}
    \fancyfoot[R]{\assignmentType\text{ }\ifdefempty{\assignmentNumber}{}{\#}\assignmentNumber}
    \fancyfoot[C]{\thepage \hspace{1pt} of \pageref{EndFrontMatter}}
    \fancyfoot[L]{\courseCode}
    \fancyhead[L]{\name}
    \fancyhead[R]{\studentNumber}
}

% Essay body page numbering %
\fancypagestyle{body}{
    \fancyhf{}
    \renewcommand{\footrulewidth}{0.1mm}
    \fancyfoot[R]{\assignmentType\text{ }\ifdefempty{\assignmentNumber}{}{\#}\assignmentNumber}
    \fancyfoot[C]{\thepage \hspace{1pt} of \pageref{LastPage}}
    \fancyfoot[L]{\courseCode}
    \fancyhead[L]{\name}
    \fancyhead[R]{\studentNumber}
}

% Page header and footers %
\fancyhf{}
\renewcommand{\footrulewidth}{0.1mm}
\fancyfoot[R]{\assignmentType\text{ }\ifdefempty{\assignmentNumber}{}{\#}\assignmentNumber}
\fancyfoot[C]{\thepage \hspace{1pt} of \pageref{LastPage}}
\fancyfoot[L]{\courseCode}
\fancyhead[L]{\name}
\fancyhead[R]{\studentNumber}

% Apply headers & footer page style %
\pagestyle{fancy}

% Assignment first page header %
\newcommand{\beginassignment}{
    % Prevent paragraph indents, this isn't an English assignment! %
    \newlength\tindent
    \setlength{\tindent}{\parindent}
    \setlength{\parindent}{0pt}

    \thispagestyle{assignment}
    \noindent
    \courseTitle \hfill \university\\
    \courseCode \hfill \semester
    \begin{center}
        \textbf{\assignmentType\text{ }\ifdefempty{\assignmentNumber}{}{\#}\assignmentNumber}\\
        \name \hspace{1pt} - \studentNumber\\
        \dueDate\\
    \end{center}
    \vspace{6pt}
    \hrule
    \vspace{1.5\headsep}
}

% Essay titlepage stuff %
\newcommand{\beginessay}{
    % Load all citations %
    \nocite{*}

    % Special numbering for front matter %
    \pagestyle{frontmatter}
    \pagenumbering{roman}

    % Titlepage stuff %
    \begin{center}
        \normalsize
        \textsc{\university}\\[5cm]
        \LARGE \textbf{\MakeUppercase{\essayTitle}}\\[0.5cm]
        \large \text{ }\essaySubtitle\text{ }\\[10cm] % blank \texts are used for empty subtitles %
        \normalsize
        \textsc{\name}\\
        \textsc{\studentNumber}\\
        \textsc{\courseCode}\\
        \textsc{\semester}\\
        \textsc{\dueDate}
    \end{center}
    \thispagestyle{empty}
    % End title page stuff %

    % Table of Contents %
    \newpage
    \renewcommand{\cftsecleader}{\cftdotfill{\cftdotsep}}
    \tableofcontents
    \newpage

    % If more than 1 table/figure show appropriate lists %
    \iftotalfigures
        \addcontentsline{toc}{section}{\listfigurename}
        \listoffigures
    \fi
    \iftotaltables
        \addcontentsline{toc}{section}{\listtablename}
        \listoftables
    \fi
    \doublespacing % Double spacing, if requested

    % Display an abstract if the abstract isn't empty %
    \ifdefempty{\essayAbstract}{}{
        \newpage
        \addcontentsline{toc}{section}{Abstract}
        \begin{abstract}
            \essayAbstract
        \end{abstract}

    }
    \label{EndFrontMatter}
    \newpage

    % Reset page numbering %
    \pagenumbering{arabic}
    \pagestyle{body}
}

% Update the bibliography command to add its self to the references
\let\oldprintbib\printbibliography
\renewcommand{\printbibliography}{
    \newpage
    \oldprintbib
    \addcontentsline{toc}{section}{References}
    \newpage
}

\newcommand{\printappendix}{
    \appendix
    \section{Appendix}
}

% Begin the document %
\begin{document}

% Generates the frontmatter for the essay %
\beginassignment


\section*{Question 1}

$
    \begin{aligned}
        \frac{\sqrt{2 \cdot 30^2}}{2} &= 21.21m
    \end{aligned}
$

$\therefore$ the positional uncertainty due to the cell resolution of 30m is 21.21m.

\section*{Question 2}

\adjincludegraphics[width=0.8\textwidth]{images/q2-1.png}
\begin{center}
    Figure 1 - Map analyzing the forest cover of the Ottawa area 2001
\end{center}

\newpage

Table 1. Confusion Matrix for the forest cover in the Ottawa area (2001)\\

\begin{tabular}{rrrrrrr}
    \hline
        OBJECTID & ClassValue & C\_1 & C\_2 & Total & U\_Accuracy & Kappa \\\hline
        1 & C\_1 & 11 & 0 & 11 & 1 & 0 \\
        2 & C\_2 & 3 & 7 & 10 & 0.7 & 0\\
        3 & Total & 14 & 7 & 21 & 0 & 0\\
        4 & P\_Accuracy & 0.785714 & 1 & 0 & 0.857143 & 0\\
        5 & Kappa & 0 & 0 & 0 & 0 & 0.709677\\
    \hline

\end{tabular}\\\\

\adjincludegraphics[width=0.8\textwidth]{images/q2-2.png}
\begin{center}
    Figure 2 - Map analyzing the forest cover of the Ottawa area in 2016
\end{center}

Table 2. Confusion Matrix for the forest cover in the Ottawa area (2016)\\

\begin{tabular}{rrrrrrr}
    \hline
        OBJECTID & ClassValue & C\_1 & C\_2 & Total & U\_Accuracy & Kappa \\\hline
        1 & C\_1 & 7 & 3 & 10 & 0.7 & 0 \\
        2 & C\_2 & 0 & 10 & 10 & 1 & 0\\
        3 & Total & 7 & 13 & 20 & 0 & 0\\
        4 & P\_Accuracy & 1 & 0.769231 & 0 & 0.85 & 0\\
        5 & Kappa & 0 & 0 & 0 & 0 & 0.7\\
    \hline
\end{tabular}\\\\

$\therefore$ The results from the 2001 confusion matrix are ever so slightly more accurate since the kappa value is ($0.709677-0.7=0.009677$) slightly higher than that of the 2016 matrix.

\section*{Question 3}

\adjincludegraphics[width=0.8\textwidth]{images/q3-1.png}
\begin{center}
    Figure 3 - Map analyzing the difference of forest coverage between 2001 and 2016 in the Ottawa region.
\end{center}

In the map a -1 represents an area with of the map with less coverage thus having a less dense forest. A 0 represents no change in the forest density. A +1 represents more coverage thus a higher density in that area.

\section*{Question 4}

The region group tool allows the us to process a large group of contiguous set of cells of the same zone type.

The link column in the attributes table links the group back to the value created during the set minus operation in question 3 (-1, 0, 1).

\newpage

Table 3. Top 10 counts from the region group attribute table\\

\begin{tabular}{rrrr}
    \hline
        OBJECTID & Value & Count & Link\\\hline
        8499 & 8499 & 993 & -1\\
        5608 & 5608 & 787 & -1\\
        4804 & 4804 & 765 & -1\\
        7855 & 7855 & 452 & -1\\
        4441 & 4441 & 222 & -1\\
        4951 & 4951 & 206 & -1\\
        7823 & 7823 & 198 & -1\\
        7135 & 7135 & 157 & -1\\
        4590 & 4590 & 153 & -1\\
        6490 & 6490 & 146 & -1\\

    \hline
\end{tabular}\\\\

\adjincludegraphics[width=0.8\textwidth]{images/q4-1.png}
\begin{center}
    Figure 4 - Map analyzing the difference of forest coverage between 2001 and 2016 in the Ottawa region. The orange regions are the top 10 regions where the density has decreased.
\end{center}

\section*{Question 5}

The mean was calculated by converting the number pixels from the count column into $m^2$. Since the layer is using 30m pixels I was able to compute a new column in the attributes table using \code{text}{(!Count!/30)*Math.pow(30, 2)}.\\

The average area computed was: $12237m^2$ for the top 10 regions.

\section*{Question 6}

For some reason when exporting the slope no data was provided and I'm not sure why. However here's the table I got:

Table 4. Average. slopes for areas over $3000m^2$\\

\begin{tabular}{rrrr}
    \hline
        OBJECTID & Value & Count & AREA & MEAN\\\hline
        1 & 8499 & 993 & -1\\

    \hline
\end{tabular}\\\\


\section*{Question 7}

The DNs in Landsat 8 are greater than those in Landsat 7 because the Landsat 8 is picking up more intense wavelengths (as seen on slide 68 of lecture 8).

\section*{Question 8}

The map shows the difference between the healthy vegetation and the degraded vegetation on the land. NVDI stands for the normalized difference vegetation index, thus showing the difference in the vegetation's health.

\adjincludegraphics[width=0.7\textwidth]{images/q8-1.png}
\begin{center}
    Figure 5 - NDVI Map for the Ottawa area in 2016
\end{center}

\end{document}
