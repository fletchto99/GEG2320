\documentclass[fleqn, 12pt]{article}

% packages %
\usepackage[includeheadfoot,headheight=15pt,margin=0.5in,left=1in,right=1in,headsep=10pt]{geometry} % page margins %
\usepackage{mathtools, amssymb} % math %
\usepackage{tabularx, multirow} % tables %
\usepackage[outputdir=.latexcache]{minted} % code %
\usepackage{graphicx} % graphics %
\usepackage{enumerate} % lists %
\usepackage{adjustbox} % images %
\usepackage[T1]{fontenc} % fonts %
\usepackage[protrusion=true,expansion=true]{microtype} % font clarity %
\usepackage{fancyhdr} % headers and footers %
\usepackage{lastpage} % reference page numbers %
\usepackage{color} % colors for code %
\usepackage{tikz} % for graphs %
\usepackage{soul} % for strikethroughout %
\usepackage{upquote} % Fix single quotes %
\usepackage{etoolbox} % Conditional checks %
\usepackage[hidelinks]{hyperref} % Hyperlinks %
\usepackage{indentfirst} % fix indentation - only for essays %
\usepackage[figure,table]{totalcount} % For counting tables and figures %
\usepackage[utf8]{inputenc} % Encode as UTF-8 %
\usepackage{biblatex} % References %
\addbibresource{references.bib} % bib source %

% Document details %
\newcommand{\university}{University of Ottawa}
\newcommand{\name}{Matthew Langlois}
\newcommand{\studentNumber}{7731813}
\newcommand{\semester}{Winter 2018}
\newcommand{\assignmentType}{Assignment}
\newcommand{\assignmentNumber}{2}
\newcommand{\dueDate}{Feb 26, 2018}
\newcommand{\courseCode}{GEG2320}
\newcommand{\courseTitle}{Introduction to Geomatics}
\newcommand{\essayTitle}{<Title>} % only for essays %
\newcommand{\essaySubtitle}{<subtitle>} % only for essays %
\newcommand{\essayAbstract}{} % Only for essays -- leave empty for no abstract %

% Center image and diagrams %
\adjustboxset*{center}

% Code settings %
\setminted{
    fontfamily=tt,
    gobble=0,
    frame=single,
    funcnamehighlighting=true,
    tabsize=4,
    obeytabs=false,
    mathescape=false
    samepage=false,
    showspaces=false,
    showtabs =false,
    texcl=false,
    breaklines=true,
}

% inline code %
\definecolor{codegray}{gray}{0.9}
\newcommand{\code}[2]{\colorbox{codegray}{\mintinline{#1}{#2}}}

% Code from tile %
\newcommand{\codefile}{\inputminted}

% Graphing stuff %
\usetikzlibrary{arrows.meta}
\usetikzlibrary{positioning}
\usetikzlibrary{matrix}
\usetikzlibrary{automata}

% Define ceiling and floor functions %
\DeclarePairedDelimiter\ceil{\lceil}{\rceil}
\DeclarePairedDelimiter\floor{\lfloor}{\rfloor}

% Create set compliment command %
\newcommand{\setcomp}[1]{{#1}^{\mathsf{c}}}

% Create logic command aliases %
\newcommand{\limplies}{\rightarrow}
\newcommand{\nequiv}{\not\equiv}
\newcommand{\liff}{\leftrightarrow}

% first page header & footer %
\fancypagestyle{assignment}{
    \fancyhf{}
    \renewcommand{\footrulewidth}{0.1mm}
    \fancyfoot[R]{\assignmentType\text{ }\assignmentNumber}
    \fancyfoot[C]{\thepage \hspace{1pt} of \pageref{LastPage}}
    \fancyfoot[L]{\courseCode\text{ }\semester}
    \renewcommand{\headrulewidth}{0mm}
}

% Frontmatter for essay page numbering%
\fancypagestyle{frontmatter}{
    \fancyhf{}
    \renewcommand{\footrulewidth}{0.1mm}
    \fancyfoot[R]{\assignmentType\text{ }\assignmentNumber}
    \fancyfoot[C]{\thepage \hspace{1pt} of \pageref{EndFrontMatter}}
    \fancyfoot[L]{\courseCode\text{ }\semester}
    \fancyhead[L]{\name}
    \fancyhead[R]{\studentNumber}
}

% Essay body page numbering %
\fancypagestyle{body}{
    \fancyhf{}
    \renewcommand{\footrulewidth}{0.1mm}
    \fancyfoot[R]{\assignmentType\text{ }\assignmentNumber}
    \fancyfoot[C]{\thepage \hspace{1pt} of \pageref{LastPage}}
    \fancyfoot[L]{\courseCode\text{ }\semester}
    \fancyhead[L]{\name}
    \fancyhead[R]{\studentNumber}
}

% Page header and footers %
\fancyhf{}
\renewcommand{\footrulewidth}{0.1mm}
\fancyfoot[R]{\assignmentType\text{ }\assignmentNumber}
\fancyfoot[C]{\thepage \hspace{1pt} of \pageref{LastPage}}
\fancyfoot[L]{\courseCode\text{ }\semester}
\fancyhead[L]{\name}
\fancyhead[R]{\studentNumber}

% Apply headers & footer page style %
\pagestyle{fancy}

% Assignment first page header %
\newcommand{\beginassignemnt}{
    % Prevent paragraph indents, this isn't an English assignment! %
    \newlength\tindent
    \setlength{\tindent}{\parindent}
    \setlength{\parindent}{0pt}

    \thispagestyle{assignment}
    \noindent
    \courseTitle \hfill \university\\
    \courseCode \hfill \semester
    \begin{center}
        \textbf{\assignmentType\text{ }\ifdefempty{\assignmentNumber}{}{\#}\assignmentNumber}\\
        \name \hspace{1pt} - \studentNumber\\
        \dueDate\\
    \end{center}
    \vspace{6pt}
    \hrule
    \vspace{1.5\headsep}
}

% Essay titlepage stuff %
\newcommand{\beginessay}{
    % Load all citations %
    \nocite{*}

    % Special numbering for front matter %
    \pagestyle{frontmatter}
    \pagenumbering{roman}

    % Titlepage stuff %
    \begin{center}
        \normalsize
        \textsc{\university}\\[5cm]
        \LARGE \textbf{\MakeUppercase{\essayTitle}}\\[0.5cm]
        \large \text{ }\essaySubtitle\text{ }\\[10cm] % blank \texts are used for empty subtitles %
        \normalsize
        \textsc{\name}\\
        \textsc{\studentNumber}\\
        \textsc{\courseCode}\\
        \textsc{\semester}\\
        \textsc{\dueDate}
    \end{center}
    \thispagestyle{empty}
    % End title page stuff %

    % Table of Contents %
    \newpage
    \tableofcontents
    \newpage

    % If more than 1 table/figure show appropriate lists %
    \iftotalfigures
        \addcontentsline{toc}{section}{\listfigurename}
        \listoffigures
    \fi
    \iftotaltables
        \addcontentsline{toc}{section}{\listtablename}
        \listoftables
    \fi

    % Display an abstract if the abstract isn't empty %
    \ifdefempty{\essayAbstract}{}{
        \newpage
        \addcontentsline{toc}{section}{Abstract}
        \begin{abstract}
            \essayAbstract
        \end{abstract}

    }
    \label{EndFrontMatter}
    \newpage

    % Reset page numbering %
    \pagenumbering{arabic}
    \pagestyle{body}
}

% Update the bibliography command to add its self to the references
\let\oldprintbib\printbibliography
\renewcommand{\printbibliography}{
    \newpage
    \oldprintbib
    \addcontentsline{toc}{section}{References}
    \newpage
}

% Begin the document %
\begin{document}

% makes the header for assignment/titlepage for essay %
% \beginessay
\beginassignemnt

\section*{Question 1}

A relationship class enforces referential integrity by ensuring related objects and their attributes are updated automatically. Rules can be set to restrict and ensure the relations formed are properly limited. For example limiting a pole has a maximum of 2 transformers on it. The referential integrity is maintained between the related classes.

\section*{Question 2}

The coordinate system being used for the census data when calculating the population density in step 6 was NAD83. NAD83 is a valid coordinate system to use when calculating census data since it is extremely accurate when used with technology thus allowing us to easily visualize the density.

\section*{Question 3}

I think the population density map is accurate. This can be shown on the map where the more densely populated areas are broken down into more regions on the map. Some of the more populated areas, such as Kanata, are a larger region. Therefore smaller but more densely populated clusters of people can be seen closer to downtown which is an accurate representation. When sorting by the smallest regions vs population we are able to highlight the most densely populated areas which I would say reflects the clusters as seen in the Ottawa area:

\adjincludegraphics[width=0.4\textwidth]{image1.png}

\section*{Question 4}

In 165 out of 271 census tracts the Male population had a greater median after-tax income. This appears to be more common in the more populated areas which tend to be the larger regions as seen below:

\adjincludegraphics[width=0.4\textwidth]{income.png}

\section*{Question 5}

In 238 out of 271 census tracts more people biked rather than walked. I believe this is the case because in the more rural areas biking would be more convenient vs walking, although more people probably drive. This pattern can be seen in the following map image:

\adjincludegraphics[width=0.4\textwidth]{bike.png}

\newpage

\section*{Question 6}

One thing I had noticed when querying the data is that there is only one region where there are more immigrants than non-immigrants:

\adjincludegraphics[width=0.35\textwidth]{more.png}

In 249 out of 271 census tracts the immigrant / non-immigrant ratio was greater than 0.65. Their average income was \$48578.

\adjincludegraphics[width=0.35\textwidth]{avgbigger.png}

In 19 out of 271 census tracts the immigrant / non-immigrant ratio was less than 0.65. Their average income was \$40335.

\adjincludegraphics[width=0.35\textwidth]{avgsmaller.png}

Overall it appears that areas with greater than 65\% non-immigrant population makes more money per year which is likely due to the fact that immigrants need to settle and find work.

\end{document}
